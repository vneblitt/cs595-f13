\documentclass{article}
\usepackage[hyphens]{url}
\usepackage{mathtools}
\usepackage{amsmath}
\usepackage{listings}
\usepackage{graphicx}
\usepackage[margin=1in]{geometry}
\usepackage{float}
\floatstyle{boxed}
\restylefloat{figure}
\lstset{basicstyle=\footnotesize, breaklines=true}
\begin{document}


\title{CS595 Intro to Web Science, Assignment \#8}
\author{Valentina Neblitt-Jones}
\date{November 14, 2013}
\maketitle



\section*{Instructions}

The goal of this project it is to use the basic recommendation principles we have learned for user-collected data. You will modify the code given to you which performs movie recommendations from the MovieLense data sets. \\

The MovieLense data sets were collected by the GroupLens Research Project at the University of Minnesota during the seven-month period from September 19th, 1997 through April 22, 1998. It is available for download from \url{http://www.grouplens.org/node/73} \\

There are three files which we will use: 

\subsection*{u.data}

u.data: 100,000 ratings by 943 users on 1,682 movies. Each user has rated at least 20 movies. Users and items are numbered consecutively from 1. The data is randomly ordered. This is a tab-separated list of user id, item id, rating, and timestamp. \\

The time stamps are unix seconds since 1/1/1970 UTC. \\

Example: 

\begin{table}[!h]
\centering
\begin{tabular}{c c c c}
user id & item id &  rating & timestamp \\
\hline
196 & 242 & 3 & 881250949 \\
186 & 302 & 3 & 891717742 \\
22 & 377 & 1 & 878887116 \\
244 & 51 & 2 & 880606923 \\
166 & 346 & 1 & 886397596 \\
298 & 474 & 4 & 884182806 \\
115 & 265 & 2 & 881171488 \\
\end{tabular}
\end{table}

\subsection*{u.item}

u.item: Information about the 1,682 movies. This is a tab separated list of movie id, movie title, release date, video release date, IMDb URL, unknown, Action, Adventure, Animation, Children's, Comedy, Crime, Documentary, Drama, Fantasy, Film-Noir, Horror, Musical, Mystery, Romance, Sci-Fi, Thriller, War, and Western. \\

The last 19 fields are the genres, a 1 indicates the movie is of that genre, a 0 indicates it is not; movies can be in several genres at once. The movie ids are the ones used in the u.data set. \\

Example: \\

\begin{table}[!h]
\centering
\begin{tabular}{c c c c c c}
movie id & movie title & release date & video release date & IMDb URL \\
\hline
161 & Top Gun (1986) & 01-Jan-1986 &  & \url{http://us.imdb.com/M/title-exact?Top%20Gun%20(1986)} &(1986) \\
%186 & 302 & 3 & 891717742 \\
%22 & 377 & 1 & 878887116 \\
\end{tabular}
\end{table}

\begin{verbatim}
161|Top Gun (1986)|01-Jan-1986||http://us.imdb.com/M/title-exact?Top\%20Gun\%20(1986)|0|1|0|0|0|0|0|0|0|0|0|0|0|0|1|0|0|0|0 
162|On Golden Pond (1981)|01-Jan-1981||http://us.imdb.com/M/title-exact?On\%20Golden\%20Pond\%20(1981)|0|0|0|0|0|0|0|0|1|0|0|0|0|0|0|0|0|0|0 
163|Return of the Pink Panther, The (1974)|01-Jan-1974||http://us.imdb.com/M/title-exact?Return\%20of\%20the\%20Pink\%20Panther,\%20The\%20(1974)|0|0|0|0|0|1|0|0|0|0|0|0|0|0| 0|0|0|0|0
\end{verbatim}

\subsection*{u.user}

u.user: Demographic information about the users. This is a tab-separate list of user id, age, gender, occupation, and zip code. \\

Example: \\

\begin{table}[!h]
\centering
\begin{tabular}{c c c c c}
user id & age &  gender & occupation & zip code \\
\hline
1 & 24 & M & technician & 85711 \\
2 & 53 & F & other & 94043 \\
2 & 23 & M & writer & 32067 \\
4 & 24 & M & technician & 43537 \\
5 & 33 & F & other & 15213 \\
\end{tabular}
\end{table}

The code for reading from the u.data and u.item files and creating recommendations is described in the book Programming Collective Intelligence (check mail for more details). You are to modify recommendations.py to answer the following questions. Each question your program answers correctly will award you 10 points. You must have the question answered completely correct; partial credit will be only awarded if your answer is very close to the correct one. Your output should clearly indicate the answers from the question you answered. Provide any relevant discussion.

\newpage
\listoftables
\lstlistoflistings

\newpage
\section*{Answers}
\renewcommand\thesubsection{\arabic{subsection}}
\subsection{What 5 movies have the highest average ratings? Show the movies and their ratings sorted by their average ratings.}

\begin{table}[!h]
\centering
\begin{tabular}{l c}
Movie Title & Average Rating \\
\hline
Santa with Muscles (1996) & 5.0  \\
Star Kid (1997) & 5.0  \\
They Made Me a Criminal (1939) & 5.0  \\
Aiqing wansui (1994) & 5.0  \\
Marlene Dietrich: Shadow and Light (1996) & 5.0  \\
A Great Day in Harlem (1994) & 5.0  \\
Entertaining Angels: The Dorothy Day Story (1996) & 5.0  \\
Someone Else's America (1995) & 5.0  \\
Prefontaine (1997) & 5.0  \\
The Saint of Fort Washington (1993) & 5.0  \\
\hline
\end{tabular}
\caption{Movies with the Highest Average Rating}
\end{table}

\begin{lstlisting}[frame=single, caption=highestavgrating.py, label=highaverage]
import datetime
import numpy
import codecs

g = open('highestaveragerating.txt', 'w')

movieratings={}
averageratings={}
movieinfo={}
topmovies={}

with open('/Users/vneblitt/Documents/cs595-f13/assignment08/dataset/u.data', 'r') as f:
	movieratinginfo = f.readlines()
	for line in movieratinginfo:
		(userid, itemid, rating, timestamp) = line.split('\t')
		if itemid in movieratings:
			movieratings[itemid].append(int(rating))
		else:
			movieratings[itemid] = [int(rating)]
f.close()

for movie in movieratings:
	ratings = movieratings[movie]
	average = numpy.mean(ratings)
	averageratings[movie] = average

#stack overflow http://stackoverflow.com/questions/613183/python-sort-a-dictionary-by-value (11/10/2013)
for movieid in sorted(averageratings, key=averageratings.get, reverse=True)[0:10]:
	topmovies[movieid] = averageratings[movieid]

with (codecs.open('/Users/vneblitt/Documents/cs595-f13/assignment08/dataset/u.item','r', 'iso-8859-1')) as h:
	moviedata = h.readlines()
	for line in moviedata:
		(movieid, movietitle) = line.split('|')[0:2]
		movieinfo[movieid] = movietitle
h.close()

for movie in topmovies:
	g.write(movieinfo[movie] + ' ' + str(topmovies[movie]) + '\n')

g.close()
\end{lstlisting}

%\begin{lstlisting}[frame=single, caption=highestavgrating.py, label=highaverage]
%\end{lstlisting}

\newpage
\subsection{What 5 movies have received the most ratings? Show the movies and the number of ratings sorted by number of ratings.}

\begin{table}[!h]
\centering
\begin{tabular}{l c}
Movie Title & No. of Ratings \\
\hline
Star Wars (1977) & 583  \\
Contact (1997) & 509  \\
Fargo (1996) & 508  \\
Return of the Jedi (1983) & 507  \\
Marlene Dietrich: Shadow and Light (1997) & 485  \\
\hline
\end{tabular}
\caption{Movies with the Most Ratings}
\end{table}

\begin{lstlisting}[frame=single, caption=mostratings.py, label=mostratings]
import codecs

g = open('mostratings.txt', 'w')

movieratings={}
countingratings={}
movieinfo={}

with open('/Users/vneblitt/Documents/cs595-f13/assignment08/dataset/u.data', 'r') as f:
	movieratinginfo = f.readlines()
	for line in movieratinginfo:
		(userid, itemid, rating, timestamp) = line.split('\t')
		if itemid in movieratings:
			movieratings[itemid].append(int(rating))
		else:
			movieratings[itemid] = [int(rating)]
f.close()

for movie in movieratings:
	ratings = movieratings[movie]
	ratingscount = len(ratings)
	countingratings[movie] = ratingscount
#print(countingratings)

with (codecs.open('/Users/vneblitt/Documents/cs595-f13/assignment08/dataset/u.item','r', 'iso-8859-1')) as h:
	moviedata = h.readlines()
	for line in moviedata:
		(movieid, movietitle) = line.split('|')[0:2]
		movieinfo[movieid] = movietitle
h.close()

#stack overflow http://stackoverflow.com/questions/613183/python-sort-a-dictionary-by-value (11/10/2013)
for movieid in sorted(countingratings, key=countingratings.get, reverse=True)[0:5]:
	g.write(movieinfo[movieid] + ' ' + str(countingratings[movieid]) + '\n')

g.close()
\end{lstlisting}

\newpage
\subsection{What 5 movies were rating the highest on average by women? Show the movies and their ratings sorted by ratings.}

\begin{table}[!h]
\centering
\begin{tabular}{l c}
Movie Title & Average Rating \\
\hline
Stripes (1981) & 5.0  \\
Mina Tannenbaum (1994) & 5.0  \\
Faster Pussycat! Kill! Kill! (1965) & 5.0  \\
Foreign Correspondent (1997) & 5.0  \\
Telling Lies in America (1996) & 5.0  \\
Maya Lin: A Strong Clear Vision (1994) & 5.0  \\
Prefontaine (1997) & 5.0  \\
Someone Else's America (1995) & 5.0  \\
Everest (1998) & 5.0  \\
Year of the Horse (1997) & 5.0  \\
The Visitors (Les Visiteurs) (1993) & 5.0 \\
\hline
\end{tabular}
\caption{Movies with the Highest Average Rating By Women}
\end{table}

\begin{lstlisting}[frame=single, caption=highestwomen.py, label=highwomen]
import codecs
import numpy

g = open('highestwomen.txt', 'w')

userinfo={}
theladies=[]
movieratings={}
averageratings={}
movieinfo={}


# Parse u.user to get the userid, age, and gender (only need userid and gender)
with open('/Users/vneblitt/Documents/cs595-f13/assignment08/dataset/u.user', 'r') as j:
	userdata = j.readlines()
	for line in userdata:
		(userid, age, gender) = line.split('|')[0:3]
		userinfo[userid] = gender

# Create a list of userids that only belong to women
for user in userinfo:
	if userinfo[user] == 'F':
		theladies.append(user)

# Create a dictionary of movies and ratings only if the user is a woman
with open('/Users/vneblitt/Documents/cs595-f13/assignment08/dataset/u.data', 'r') as f:
	movieratinginfo = f.readlines()
	for line in movieratinginfo:
		(userid, itemid, rating, timestamp) = line.split('\t')
		if userid in theladies:
			if itemid in movieratings:
				movieratings[itemid].append(int(rating))
			else:
				movieratings[itemid] = [int(rating)]
f.close()

# Calculate the average rating for each movie
for movie in movieratings:
	ratings = movieratings[movie]
	average = numpy.mean(ratings)
	averageratings[movie] = average

with (codecs.open('/Users/vneblitt/Documents/cs595-f13/assignment08/dataset/u.item','r', 'iso-8859-1')) as h:
	moviedata = h.readlines()
	for line in moviedata:
		(movieid, movietitle) = line.split('|')[0:2]
		movieinfo[movieid] = movietitle
h.close()

#stack overflow http://stackoverflow.com/questions/613183/python-sort-a-dictionary-by-value (11/10/2013)
for movieid in sorted(averageratings, key=averageratings.get, reverse=True)[0:11]:
	g.write(movieinfo[movieid] + ' ' + str(averageratings[movieid]) + '\n')

g.close()
\end{lstlisting}

\newpage
\subsection{What 5 movies were rating the highest on average by men? Show the movies and their ratings sorted by ratings.}

\begin{table}[!h]
\centering
\begin{tabular}{l c}
Movie Title & Average Rating \\
\hline
A Great Day in Harlem (1994) & 5.0 \\
Delta of Venus (1994) & 5.0 \\
Love Serenade (1996) & 5.0 \\
They Made Me a Criminal (1939) & 5.0 \\
Hugo Pool (1997) & 5.0 \\
A Letter From Death Row (1998) & 5.0 \\
Prefontaine (1997) & 5.0 \\
Santa with Muscles (1996) & 5.0 \\
The Saint of Fort Washington (1993) & 5.0 \\
Star Kid (1997) & 5.0 \\
Aiqing wansui (1994) & 5.0 \\
The Leading Man (1996) & 5.0 \\
The Quiet Room (1996) & 5.0 \\
Marlene Dietrich: Shadow and Light (1996)  & 5.0 \\
Little City (1998) & 5.0 \\
Entertaining Angels: The Dorothy Day Story (1996) & 5.0 \\
\hline
\end{tabular}
\caption{Movies with the Highest Average Rating By Men}
\end{table}

\begin{lstlisting}[frame=single, caption=highestmen.py, label=highmen]
import codecs
import numpy

g = open('highestmen.txt', 'w')

userinfo={}
theguys=[]
movieratings={}
averageratings={}
movieinfo={}


# Parse u.user to get the userid, age, and gender (only need userid and gender)
with open('/Users/vneblitt/Documents/cs595-f13/assignment08/dataset/u.user', 'r') as j:
	userdata = j.readlines()
	for line in userdata:
		(userid, age, gender) = line.split('|')[0:3]
		userinfo[userid] = gender

# Create a list of userids that only belong to women
for user in userinfo:
	if userinfo[user] == 'M':
		theguys.append(user)

# Create a dictionary of movies and ratings only if the user is a man
with open('/Users/vneblitt/Documents/cs595-f13/assignment08/dataset/u.data', 'r') as f:
	movieratinginfo = f.readlines()
	for line in movieratinginfo:
		(userid, itemid, rating, timestamp) = line.split('\t')
		if userid in theguys:
			if itemid in movieratings:
				movieratings[itemid].append(int(rating))
			else:
				movieratings[itemid] = [int(rating)]
f.close()

# Calculate the average rating for each movie
for movie in movieratings:
	ratings = movieratings[movie]
	average = numpy.mean(ratings)
	averageratings[movie] = average

with (codecs.open('/Users/vneblitt/Documents/cs595-f13/assignment08/dataset/u.item','r', 'iso-8859-1')) as h:
	moviedata = h.readlines()
	for line in moviedata:
		(movieid, movietitle) = line.split('|')[0:2]
		movieinfo[movieid] = movietitle
h.close()

#stack overflow http://stackoverflow.com/questions/613183/python-sort-a-dictionary-by-value (11/10/2013)
for movieid in sorted(averageratings, key=averageratings.get, reverse=True)[0:16]:
	g.write(movieinfo[movieid] + ' ' + str(averageratings[movieid]) + '\n')

g.close()
\end{lstlisting}

%\begin{lstlisting}[frame=single, caption=highestavgrating.py, label=highaverage]
%\end{lstlisting}

\newpage
\subsection{What movie received ratings most like Top Gun? Which movie received ratings that were least like Top Gun (negative correlation)?}

\newpage
\subsection{Which 5 raters rated the most films? Show the raters' IDs and the number of films each rated.}

\begin{table}[!h]
\centering
\begin{tabular}{l c}
Rater ID & No. of Ratings \\
\hline
405 & 737  \\
655 & 685  \\
13 & 636  \\
450 & 540  \\
276 & 518  \\
\hline
\end{tabular}
\caption{Raters Who Rated the Most Films}
\end{table}

\begin{lstlisting}[frame=single, caption=prolificraters.py, label=prolificraters]
g = open('prolificraters.txt', 'w')

movieraters={}
countingratings={}
topraters={}

with open('/Users/vneblitt/Documents/cs595-f13/assignment08/dataset/u.data', 'r') as f:
	movieratinginfo = f.readlines()
	for line in movieratinginfo:
		(userid, itemid, rating, timestamp) = line.split('\t')
		if userid in movieraters:
			movieraters[userid].append(int(rating))
		else:
			movieraters[userid] = [int(rating)]
f.close()
#print(movieraters)

for rater in movieraters:
	ratings = movieraters[rater]
	ratingscount = len(ratings)
	countingratings[rater] = ratingscount
#print(countingratings)

g.write('rater id, ' + 'number of ratings' + '\n')

#stack overflow http://stackoverflow.com/questions/613183/python-sort-a-dictionary-by-value (11/10/2013)
for rater in sorted(countingratings, key=countingratings.get, reverse=True)[0:5]:
	topraters[rater] = countingratings[rater]
	g.write(rater + ', ' + str(countingratings[rater]) + '\n')

g.close()
\end{lstlisting}

\newpage
\subsection{Which 5 raters most agreed with each other? Show the raters' IDs and Pearson's r, sorted by r.}

\newpage
\subsection{Which 5 raters most disagreed with each other (negative correlation)? Show the raters' IDs and Pearson's r, sorted by r.}

\newpage
\subsection{What movie was rated highest on average by men over 40? By men under 40?}

\begin{table}[!h]
\centering
\begin{tabular}{l c}
Movie Title & Average Rating \\
\hline
The World of Apu (Apur Sansar) (1959) & 5.0 \\
Indian Summer (1996) & 5.0 \\
Star Kid (1997) & 5.0 \\
Hearts and Minds (1996) & 5.0 \\
Marlene Dietrich: Shadow and Light (1996)  & 5.0 \\
Aparajito (1956) & 5.0 \\
A Great Day in Harlem (1994) & 5.0 \\
Strawberry and Chocolate (Fresa y chocolate) (1993) & 5.0 \\
Unstrung Heroes (1995) & 5.0 \\
Prefontaine (1997) & 5.0 \\
Faithful (1996) & 5.0 \\
Late Bloomers (1996) & 5.0 \\
Double Happiness (1994) & 5.0 \\
The Leading Man (1996) & 5.0 \\
They Made Me a Criminal (1939) & 5.0 \\
Little City (1998) & 5.0 \\
The Little Princess (1939) & 5.0 \\
Grateful Dead (1995) & 5.0 \\
Solo (1996) & 5.0 \\
Two or Three Things I Know About Her (1966) & 5.0 \\
Rendezvous in Paris (Les Rendez-vous de Paris) (1995) & 5.0 \\
Spice World (1997) & 5.0 \\
Poison Ivy II (1995) & 5.0 \\
Boxing Helena (1993) & 5.0 \\
Ace Ventura: When Nature Calls (1995) & 5.0 \\
\hline
\end{tabular}
\caption{Movies with the Highest Average Rating By Men Over 40}
\end{table}

\begin{lstlisting}[frame=single, caption=highestmenover40.py, label=highestmenover40]
import codecs
import numpy

g = open('highestmenover40.txt', 'w')

theguys=[]
movieratings={}
averageratings={}
movieinfo={}


# Parse u.user to get the userid, age, and gender and create a list of userids that only contain men over 40
with open('/Users/vneblitt/Documents/cs595-f13/assignment08/dataset/u.user', 'r') as j:
	userdata = j.readlines()
	for line in userdata:
		(userid, age, gender) = line.split('|')[0:3]
		if gender == 'M':
			if int(age) > 40:
				theguys.append(userid)

# Create a dictionary of movies and ratings only if the user is a men over 40
with open('/Users/vneblitt/Documents/cs595-f13/assignment08/dataset/u.data', 'r') as f:
	movieratinginfo = f.readlines()
	for line in movieratinginfo:
		(userid, itemid, rating, timestamp) = line.split('\t')
		if userid in theguys:
			if itemid in movieratings:
				movieratings[itemid].append(int(rating))
			else:
				movieratings[itemid] = [int(rating)]
f.close()

# Calculate the average rating for each movie
for movie in movieratings:
	ratings = movieratings[movie]
	average = numpy.mean(ratings)
	averageratings[movie] = average

with (codecs.open('/Users/vneblitt/Documents/cs595-f13/assignment08/dataset/u.item','r', 'iso-8859-1')) as h:
	moviedata = h.readlines()
	for line in moviedata:
		(movieid, movietitle) = line.split('|')[0:2]
		movieinfo[movieid] = movietitle
h.close()

#stack overflow http://stackoverflow.com/questions/613183/python-sort-a-dictionary-by-value (11/10/2013)
for movieid in sorted(averageratings, key=averageratings.get, reverse=True)[0:25]:
	g.write(movieinfo[movieid] + ' ' + str(averageratings[movieid]) + '\n')

g.close()
\end{lstlisting}

\begin{table}[!h]
\centering
\begin{tabular}{l c}
Movie Title & Average Rating \\
\hline
Angel Baby (1995) & 5.0 \\
Crossfire (1947) & 5.0 \\
Love Serenade (1996) & 5.0 \\
The Saint of Fort Washington (1993) & 5.0 \\
A Perfect Candidate (1996) & 5.0 \\
Delta of Venus (1994) & 5.0 \\
Entertaining Angels: The Dorothy Day Story (1996) & 5.0 \\
The Leading Man (1996) & 5.0 \\
Star Kid (1997) & 5.0 \\
Aiqing wansui (1994) & 5.0 \\
Santa with Muscles (1996) & 5.0 \\
The Magic Hour (1998) & 5.0 \\
The Quiet Room (1996) & 5.0 \\
Hugo Pool (1997) & 5.0 \\
Maya Lin: A Strong Clear Vision (1994) & 5.0 \\
Prefontaine (1997) & 5.0 \\
A Letter From Death Row (1998) & 5.0 \\
Love in the Afternoon (1957) & 5.0 \\
\hline
\end{tabular}
\caption{Movies with the Highest Average Rating By Men Under 40}
\end{table}

\begin{lstlisting}[frame=single, caption=highestmenunder40.py, label=highestmenunder40]
import codecs
import numpy

g = open('highestmenunder40.txt', 'w')

theguys=[]
movieratings={}
averageratings={}
movieinfo={}


# Parse u.user to get the userid, age, and gender and create a list of userids that only contain men under 40
with open('/Users/vneblitt/Documents/cs595-f13/assignment08/dataset/u.user', 'r') as j:
	userdata = j.readlines()
	for line in userdata:
		(userid, age, gender) = line.split('|')[0:3]
		if gender == 'M':
			if int(age) < 40:
				theguys.append(userid)

# Create a dictionary of movies and ratings only if the user is a men under 40
with open('/Users/vneblitt/Documents/cs595-f13/assignment08/dataset/u.data', 'r') as f:
	movieratinginfo = f.readlines()
	for line in movieratinginfo:
		(userid, itemid, rating, timestamp) = line.split('\t')
		if userid in theguys:
			if itemid in movieratings:
				movieratings[itemid].append(int(rating))
			else:
				movieratings[itemid] = [int(rating)]
f.close()

# Calculate the average rating for each movie
for movie in movieratings:
	ratings = movieratings[movie]
	average = numpy.mean(ratings)
	averageratings[movie] = average

with (codecs.open('/Users/vneblitt/Documents/cs595-f13/assignment08/dataset/u.item','r', 'iso-8859-1')) as h:
	moviedata = h.readlines()
	for line in moviedata:
		(movieid, movietitle) = line.split('|')[0:2]
		movieinfo[movieid] = movietitle
h.close()

#stack overflow http://stackoverflow.com/questions/613183/python-sort-a-dictionary-by-value (11/10/2013)
for movieid in sorted(averageratings, key=averageratings.get, reverse=True)[0:18]:
	g.write(movieinfo[movieid] + ' ' + str(averageratings[movieid]) + '\n')

g.close()
\end{lstlisting}

\newpage
\subsection{What movie was rated highest on average by women over 40? By women under 40?}

\begin{table}[!h]
\centering
\begin{tabular}{l c}
Movie Title & Average Rating \\
\hline
Balto (1995) & 5.0 \\
Pocahontas (1995) & 5.0 \\
A Grand Day Out (1992) & 5.0 \\
Mary Shelley's Frankenstein (1994) & 5.0 \\
Ma vie en rose (My Life in Pink) (1997) & 5.0 \\
Bride of Frankenstein (1935) & 5.0 \\
Shall We Dance? (1937) & 5.0 \\
The Visitors (Les Visiteurs) (1993) & 5.0 \\
The Great Dictator (1940) & 5.0 \\
A Letter From Death Row (1998) & 5.0 \\
The Wrong Trousers (1993) & 5.0 \\
In the Bleak Midwinter (1995) & 5.0 \\
Best Men (1997) & 5.0 \\
Safe (1995) & 5.0 \\
Funny Face (1957) & 5.0 \\
Tombstone (1993) & 5.0 \\
Angel Baby (1995) & 5.0 \\
The Band Wagon (1953) & 5.0 \\
The Quest (1996) & 5.0 \\
The Nightmare Before Christmas (1993) & 5.0 \\
Gold Diggers: The Secret of Bear Mountain (1995) & 5.0 \\
Shallow Grave (1994) & 5.0 \\
Foreign Correspondent (1940) & 5.0 \\
Top Hat (1935) & 5.0 \\
Mina Tannenbaum (1994) & 5.0 \\
Swept from the Sea (1997) & 5.0 \\
\hline
\end{tabular}
\caption{Movies with the Highest Average Rating By Women Over 40}
\end{table}

\begin{lstlisting}[frame=single, caption=highestwomenover40.py, label=highestwomenover40]
import codecs
import numpy

g = open('highestwomenover40.txt', 'w')

theladies=[]
movieratings={}
averageratings={}
movieinfo={}


# Parse u.user to get the userid, age, and gender and create a list of userids that only contain women over 40
with open('/Users/vneblitt/Documents/cs595-f13/assignment08/dataset/u.user', 'r') as j:
	userdata = j.readlines()
	for line in userdata:
		(userid, age, gender) = line.split('|')[0:3]
		if gender == 'F':
			if int(age) > 40:
				theladies.append(userid)

# Create a dictionary of movies and ratings only if the user is a woman over 40
with open('/Users/vneblitt/Documents/cs595-f13/assignment08/dataset/u.data', 'r') as f:
	movieratinginfo = f.readlines()
	for line in movieratinginfo:
		(userid, itemid, rating, timestamp) = line.split('\t')
		if userid in theladies:
			if itemid in movieratings:
				movieratings[itemid].append(int(rating))
			else:
				movieratings[itemid] = [int(rating)]
f.close()

# Calculate the average rating for each movie
for movie in movieratings:
	ratings = movieratings[movie]
	average = numpy.mean(ratings)
	averageratings[movie] = average

with (codecs.open('/Users/vneblitt/Documents/cs595-f13/assignment08/dataset/u.item','r', 'iso-8859-1')) as h:
	moviedata = h.readlines()
	for line in moviedata:
		(movieid, movietitle) = line.split('|')[0:2]
		movieinfo[movieid] = movietitle
h.close()

#stack overflow http://stackoverflow.com/questions/613183/python-sort-a-dictionary-by-value (11/10/2013)
for movieid in sorted(averageratings, key=averageratings.get, reverse=True)[0:26]:
	g.write(movieinfo[movieid] + ' ' + str(averageratings[movieid]) + '\n')

g.close()
\end{lstlisting}

\begin{table}[!h]
\centering
\begin{tabular}{l c}
Movie Title & Average Rating \\
\hline
Grace of My Heart (1996) & 5.0 \\
Don't Be a Menace to South Central While Drinking Your Juice in the Hood (1996) & 5.0 \\
The Umbrellas of Cherbourg (Les Parapluies de Cherbourg) (1964) & 5.0 \\
Telling Lies in America (1997) & 5.0 \\
Year of the Horse (1997) & 5.0 \\
Stripes (1981) & 5.0 \\
Faster Pussycat! Kill! Kill! (1965) & 5.0 \\
Heaven's Prisoners (1996) & 5.0 \\
Prefontaine (1997) & 5.0 \\
Mina Tannenbaum (1994) & 5.0 \\
Maya Lin: A Strong Clear Vision (1994) & 5.0 \\
The Horseman on the Roof (Le Hussard sur le toit) (1995) & 5.0 \\
Someone Else's America (1995) & 5.0 \\
Everest (1998) & 5.0 \\
The Wedding Gift (1994) & 5.0 \\
Backbeat (1993) & 5.0 \\
Nico Icon (1995) & 5.0 \\
\hline
\end{tabular}
\caption{Movies with the Highest Average Rating By Women Under 40}
\end{table}

\begin{lstlisting}[frame=single, caption=highestwomenunder40.py, label=highwomenunder40]
import codecs
import numpy

g = open('highestwomenunder40.txt', 'w')

theladies=[]
movieratings={}
averageratings={}
movieinfo={}


# Parse u.user to get the userid, age, and gender and create a list of userids that only contain women under 40
with open('/Users/vneblitt/Documents/cs595-f13/assignment08/dataset/u.user', 'r') as j:
	userdata = j.readlines()
	for line in userdata:
		(userid, age, gender) = line.split('|')[0:3]
		if gender == 'F':
			if int(age) < 40:
				theladies.append(userid)

# Create a dictionary of movies and ratings only if the user is a woman under 40
with open('/Users/vneblitt/Documents/cs595-f13/assignment08/dataset/u.data', 'r') as f:
	movieratinginfo = f.readlines()
	for line in movieratinginfo:
		(userid, itemid, rating, timestamp) = line.split('\t')
		if userid in theladies:
			if itemid in movieratings:
				movieratings[itemid].append(int(rating))
			else:
				movieratings[itemid] = [int(rating)]
f.close()

# Calculate the average rating for each movie
for movie in movieratings:
	ratings = movieratings[movie]
	average = numpy.mean(ratings)
	averageratings[movie] = average

with (codecs.open('/Users/vneblitt/Documents/cs595-f13/assignment08/dataset/u.item','r', 'iso-8859-1')) as h:
	moviedata = h.readlines()
	for line in moviedata:
		(movieid, movietitle) = line.split('|')[0:2]
		movieinfo[movieid] = movietitle
h.close()

#stack overflow http://stackoverflow.com/questions/613183/python-sort-a-dictionary-by-value (11/10/2013)
for movieid in sorted(averageratings, key=averageratings.get, reverse=True)[0:17]:
	g.write(movieinfo[movieid] + ' ' + str(averageratings[movieid]) + '\n')

g.close()
\end{lstlisting}

\newpage

\section*{Resources}

%\begin{itemize}
%\item Bostock, Michael. Data-Driven Documents. \url{http://d3js.org/}
%\item Bostock, Michael. Labeled Force Layout. \url{http://bl.ocks.org/mbostock/950642}
%\item Bostock, Michael. Selections. \url{https://github.com/mbostock/d3/wiki/Selections#wiki-on}
%\item Bostock, Michael. mbostock/d3. \url{https://github.com/mbostock/d3}
%\item Bostock, Michael. mbostock/d3: API Reference. \url{https://github.com/mbostock/d3/wiki/API-Reference}
%\item Murray, Scott. About these tutorial. \url{http://alignedleft.com/tutorials/d3/about}
%\item Murray, Scott. Interactive Data Visualization: An Introduction to Designing with D3. Via Safari Books Online
%\item NetworkX developer team. NetworkX. \url{http://networkx.github.io/}
%\item NetworkX developer team. NetworkX: edge\_betweenness\_centrality. \url{http://networkx.github.io/documentation/latest/reference/generated/networkx.algorithms.centrality.edge_betweenness_centrality.html#networkx.algorithms.centrality.edge_betweenness_centrality}
%\item NetworkX developer team. NetworkX: Graph - Undirected graphs with self loops. \url{http://networkx.github.io/documentation/latest/reference/classes.graph.html#networkx.Graph}
%\item NetworkX developer team. NetworkX: node\_link\_data. \url{http://networkx.lanl.gov/reference/generated/networkx.readwrite.json_graph.node_link_data.html#networkx.readwrite.json_graph.node_link_data}
%\item NetworkX developer team. NetworkX: node\_link\_graph. \url{http://networkx.lanl.gov/reference/generated/networkx.readwrite.json_graph.node_link_graph.html#networkx.readwrite.json_graph.node_link_graph}
%\item NetworkX developer team. NetworkX: number\_connected\_components. \url{http://networkx.github.io/documentation/latest/reference/generated/networkx.algorithms.components.connected.number_connected_components.html#networkx.algorithms.components.connected.number_connected_components}
%\item PyMOTW. json: JavaScript Objection Notation Serializer. \url{http://pymotw.com/2/json/}
%\item Python.org. json: JSON encoder and decoder. \url{http://docs.python.org/3.3/library/json.html}
%\item Python.org. print: Data pretty printer. \url{http://docs.python.org/2/library/pprint.html}
%\item Stack Overflow. Parsing values from a JSON file in Python. \url{http://stackoverflow.com/questions/2835559/parsing-values-from-a-json-file-in-python}
%\item Stack Overflow. Skip the first couple of lines while reading lines in Python file. \url{http://stackoverflow.com/questions/9578580/skip-first-couple-of-lines-while-reading-lines-in-python-file}
%\end{itemize}

\end{document}