\documentclass{article}
\usepackage{url}
\usepackage{mathtools}
\usepackage{amsmath}
\usepackage{listings}
\usepackage{graphicx}
\usepackage[margin=1in]{geometry}
\usepackage{float}
\floatstyle{boxed}
\restylefloat{figure}
\lstset{breaklines=true}
\begin{document}

\title{CS595 Intro to Web Science, Assignment \#2}
\author{Valentina Neblitt-Jones}
\date{September 26, 2013}
\maketitle

\section{Link Extraction from Twitter}
\textbf {Write a Python program that extracts 1000 unique links from Twitter. You might want to take a look at: \url{http://thomassileo.com/blog/2013/01/25/using-twitter-rest-api-v1-dot-1-with-python/}. But there are many other similar resources available on the web. Note that only Twitter API 1.1 is currently available; version 1 code will no longer work. Also note that you need to verify that the final target URI (i.e., the one that responds with a 200) is unique. You could have different shortened URIs for www.cnn.com. You might want to use the search feature (Figure \ref{turtle}) to find URIs, or you can pull them from the feed of someone famous (e.g., Tim O'Reilly). Hold on to this collection. We'll use it later through the semester.}


\subsection*{The Files}
Files Used to Complete Q1

\begin{enumerate}
\item TwitterLink.py - Gathering tweets from 27 users
\item tweetlink.txt - File created by TwitterLink.py - contains URIs retrieved from tweet
\item UnpackURIs.py - Unshorten links from tweets
\item unpackedURLs.txt - File created by UnpackURIs.py - contains full URIs
\item DedupeURIs.py - Remove duplicate URIs
\item uniqueURIs.txt - File created by DedupeURIs - contains unique URIs
\end{enumerate}

\subsection*{Tweeters}
I used 27 tweeters. I could have used less by continuing to loop through some of the more prolific tweeters' timelines, but I thought more tweeters would provide a better variety of links.

\begin{itemize}
\item Michael Moore
\item Rachel Maddow
\item New York Times
\item Sesame Street
\item The Daily Show
\item Washington Post
\item Barack Obama
\item Cory Booker
\item Joel Spolsky
\item Governor Christie
\item Planned Parenthood
\item National Zoo
\item United Nations
\item Washington City Paper
\item Entertainment Weekly
\item TMZ
\item NPR
\item Virginia Tech News
\item New York Public Library
\item Library of Congress
\item Chicago Sun Times
\item Chicago Tribune
\item USA.gov
\item Harvard University
\item NFL
\item NPR News
\item Neil deGrasse Tyson
\end{itemize}


%\begin{lstlisting}
%\end{lstlisting}

\subsection*{The Execution}




\begin{figure}[H]
\centering
%\includegraphics[scale=0.50]{}
\caption{}
\label{turtle}
\end{figure}

\newpage

\section{TimeMaps Exercise}
\textbf{Download the TimeMaps for each of the target URIs. We'll use the ODU Memento Aggregator, so for example:}

\textbf{URI-R = \url{http://www.cs.odu.edu}}

\textbf{URI-T = \url{http://mementoproxy.cs.odu.edu/aggr/timemap/link/http://www.cs.odu.edu/}}

\textbf{Create a histogram of URIs vs. number of Mementos (as computed from the TimeMaps). For example, 100 URIs with 0 Mementos, 300 URIs with 1 Memento, 400 URIs with 2 Mementos, etc.}

\subsection*{The Files}
Files Used to Complete Q2

\begin{enumerate}
\item uniqueURIs2.txt - Copy of file created by DedupeURIs - contains unique URIs
\item GetTimemaps.py - Loops through unique URIs to aggregate and count the number of mementos for each
\item timeMapResults.txt - Output of the URIs and number of mementos found for each
\item TimeMapAnalysis.xslx - Excel workbook that has the raw data, processed data and the histogram
\end{enumerate}



%\begin{enumerate}
%\item school
%\item timeout
%\item uri
%\end{enumerate}


\newpage

\section{Carbon Date Exercise}
\textbf{Estimate the age of each of the 1000 URIs using the "Carbon Date" tool: \url{http://ws-dl.blogspot.com/2013/04/2013-04-19-carbon-dating-web.html}. Note you'll have to download and install; don't try to use the web service. For URIs that have \textgreater  0 Mementos and an estimated date, create a graph with age (in days) on one axis and number of Mementos on the other.}

\begin{verbatim}

\end{verbatim}



%\begin{itemize}
%\item 
%\item 
%\item 
%\item 
%\item 
%\item 
%\end{itemize}


\end{document}