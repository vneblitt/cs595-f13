\documentclass{article}
\usepackage{url}
\usepackage{mathtools}
\usepackage{amsmath}
\usepackage{listings}
\usepackage{graphicx}
\usepackage[margin=1in]{geometry}
\usepackage{float}
\floatstyle{boxed}
\restylefloat{figure}
\lstset{breaklines=true}
\begin{document}

\title{CS595 Intro to Web Science, Assignment \#2}
\author{Valentina Neblitt-Jones}
\date{September 19, 2013}
\maketitle

\section{Link Extraction from Twitter Exercise}
\textbf {Write a Python program that extracts 1000 unique links from Twitter. You might want to take a look at: \url{http://thomassileo.com/blog/2013/01/25/using-twitter-rest-api-v1-dot-1-with-python/}. But there are many other similar resources available on the web. Note that only Twitter API 1.1 is currently available; version 1 code will no longer work. Also note that you need to verify that the final target URI (i.e., the one that responds with a 200) is unique. You could have different shortened URIs for www.cnn.com. You might want to use the search feature (Figure \ref{turtle}) to find URIs, or you can pull them from the feed of someone famous (e.g., Tim O'Reilly). Hold on to this collection. We'll use it later through the semester.}


\subsection*{The Code}

\begin{lstlisting}

# -*- encoding: utf-8 -*-
from __future__ import unicode_literals
import requests
from requests_oauthlib import OAuth1
from urlparse import parse_qs

REQUEST_TOKEN_URL = "https://api.twitter.com/oauth/request_token"
AUTHORIZE_URL = "https://api.twitter.com/oauth/authorize?oauth_token="
ACCESS_TOKEN_URL = "https://api.twitter.com/oauth/access_token"

CONSUMER_KEY = "YOUR_CONSUMER_KEY"
CONSUMER_SECRET = "YOUR CONSUMER_SECRET"

OAUTH_TOKEN = ""
OAUTH_TOKEN_SECRET = ""


def setup_oauth():
    """Authorize your app via identifier."""
    # Request token
    oauth = OAuth1(CONSUMER_KEY, client_secret=CONSUMER_SECRET)
    r = requests.post(url=REQUEST_TOKEN_URL, auth=oauth)
    credentials = parse_qs(r.content)
    
    resource_owner_key = credentials.get('oauth_token')[0]
    resource_owner_secret = credentials.get('oauth_token_secret')[0]
    
    # Authorize
    authorize_url = AUTHORIZE_URL + resource_owner_key
    print 'Please go here and authorize: ' + authorize_url
    
    verifier = raw_input('Please input the verifier: ')
    oauth = OAuth1(CONSUMER_KEY,
                   client_secret=CONSUMER_SECRET,
                   resource_owner_key=resource_owner_key,
                   resource_owner_secret=resource_owner_secret,
                   verifier=verifier)
    
    # Finally, Obtain the Access Token
    r = requests.post(url=ACCESS_TOKEN_URL, auth=oauth)
    credentials = parse_qs(r.content)
    token = credentials.get('oauth_token')[0]
    secret = credentials.get('oauth_token_secret')[0]
    
    return token, secret


def get_oauth():
    oauth = OAuth1(CONSUMER_KEY,
                   client_secret=CONSUMER_SECRET,
                   resource_owner_key=OAUTH_TOKEN,
                   resource_owner_secret=OAUTH_TOKEN_SECRET)
    return oauth

if __name__ == "__main__":
    if not OAUTH_TOKEN:
        token, secret = setup_oauth()
        print "OAUTH_TOKEN: " + token
        print "OAUTH_TOKEN_SECRET: " + secret
        print
    else:
        oauth = get_oauth()
        r = requests.get(url="https://api.twitter.com/1.1/statuses/mentions_timeline.json", auth=oauth)
        print r.json()

\end{lstlisting}

\subsection*{The Execution}




\begin{figure}[H]
\centering
%\includegraphics[scale=0.50]{}
\caption{}
\label{turtle}
\end{figure}

\newpage

\section{TimeMaps Exercise}
\textbf{Download the TimeMaps for each of the target URIs. We'll use the ODU Memento Aggregator, so for example:}

\textbf{URI-R = \url{http://www.cs.odu.edu}}

\textbf{URI-T = \url{http://mementoproxy.cs.odu.edu/aggr/timemap/link/http://www.cs.odu.edu/}}

\textbf{Create a histogram of URIs vs. number of Mementos (as computed from the TimeMaps). For example, 100 URIs with 0 Mementos, 300 URIs with 1 Memento, 400 URIs with 2 Mementos, etc.}




%\begin{enumerate}
%\item school
%\item timeout
%\item uri
%\end{enumerate}


\newpage

\section{Carbon Date Exercise}
\textbf{Estimate the age of each of the 1000 URIs using the "Carbon Date" tool: \url{http://ws-dl.blogspot.com/2013/04/2013-04-19-carbon-dating-web.html}. Note you'll have to download and install; don't try to use the web service. For URIs that have \textgreater  0 Mementos and an estimated date, create a graph with age (in days) on one axis and number of Mementos on the other.}

\begin{verbatim}

\end{verbatim}



%\begin{itemize}
%\item 
%\item 
%\item 
%\item 
%\item 
%\item 
%\end{itemize}


\end{document}